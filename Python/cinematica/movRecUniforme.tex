\documentclass[a4paper,10p]{article}
\usepackage[utf8]{inputenc}
\usepackage[spanish]{babel}
\usepackage[T1]{fontenc}

\begin{document}

Sabemos que para poder describir la trayectoria de una particula, es necesario conocer su posicion a traves del tiempo, es decir, la posición como función del tiempo, $x(t)$. Con esto podemos obtener por medio de condiciones iniciales todos los puntos de la trayectoria. 
\newline

Consideremos el caso discreto, consideremos el intervalo de tiempo $t \in [t_0,t_1]$, si lo particionamos en $N$ partes iguales de tamaño $\Delta t = \frac{t_1-t_0}{N}$, de modo que   podemos escribir el tiempo $t_n$ como:

\begin{eqnarray*}
    t_n = t_0 + n\Delta t.    
\end{eqnarray*}

Por lo que si llamamos $x_n(t_n)$ a la posicion en la que se encuentra la particula en el tiempo $t_n$. Entonces solo basta encontrar una expresión para describir la misma posicion en pero en un tiempo posterior, $x_n(t_n + \Delta t)$ en terminos de $x_n(t_n)$, con $\Delta t$ lo suficientemente pequeño $\Delta t \rightarrow 0$, para poder hacer una aproximación en terminos de una serie de Taylor al rededor de $t_n$.
\newline

\begin{eqnarray*}
x_n(t_n + \Delta t) = \sum_{n=0}^{\infty} \frac{\Delta t^n}{n!} \left. \frac{d^{(n)} }{d t_n^{(n)}}  x_n(t_n) \right|_{\Delta t}    
\end{eqnarray*}

Si despresiamos terminos de segundo orden $O(t_n^2)$, entonces tenenmos,

\begin{eqnarray*}
    x_n(t_n + \Delta t) \approx x_n(t_n) +  \left. \frac{d x_n}{d t_n} \right|_{\Delta t} \Delta t.
\end{eqnarray*}
    
Observación,

\begin{eqnarray*}
    t_n + \Delta t &=& t_0 + n\Delta t + \Delta t,\\
                   &=& t_0 + (n+1)\Delta t,\\
                   &=& t_{n+1}.
\end{eqnarray*}

Por lo tanto podemos escribir,

\begin{equation}
    x_n(t_{n+1}) = x_n(t_n) +  v(t_n) \Delta t.
\end{equation}

Donde $v(t_n) = \left. \frac{d x_n}{d t_n} \right|_{\Delta t}$ la podemos definir como la velocidad de la particula en el tiempo $t_n$. De modo que solo basta con especificar la dependencia tiene la velocidad, en general $v(t_n,x_n)$ y la condición inicial para determinar la trayectoria de la particula al menos a primer orden.

















   























\end{document}